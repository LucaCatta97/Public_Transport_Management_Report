\chapter{Introduction}
The objective of this project course is to re-design a bus public transport service for one of the new biggest working poles in the Pavia area: the business park in Stradella.

This rural territory is living an intense sub-urban commercial re-development process from the last 5 years thanks to both its high availability of commercial buildable lands and its proximity to mobility infrastructure. In fact, the area not only offers the direct access to the A21 “Wine highway”, many capillary freeways and the railway infrastructure network, but also has a strategic proximity to many important cities such as Milan (60 km) ,  Pavia (20 km), Cremona (50 km),  Alessandria (60 km), Parma (100 km), Turin (150km) and the Genova harbor (115 km).

This higher and higher industrial activities development needs more and more manpower availability, and thus creating a huge accessibility demand for the employees. Estimated around for 2000 people per day, this labor mobility needs has mainly two sources of demand fragmentations: a temporal and spatial one. The first feature is given by the different working time of the operating companies, which could be organized according to a full-day working time – 24 hours per day by 3 working shifts at 6, 14 and 22. – or by a more traditional daily-time working schedule from 8.30 to 17.30  Instead, the second fragmentation feature is given by the manpower residential spatial location, which is not limited to the Stradella area – which count only 11’600 inhabitants- but also on the surrounding cities of Broni, Voghera, Pavia, Casteggio and Castel San Giovanni.

The combination of these two fragmentation demand sources with an actual absence of a tailored public transport service able to answer to the new business park mobility needs, could easily lead to car-oriented employees modal choice: unsustainable on the long period both on the environmental and traffic side. Aware of this possible oncoming problem, the local Public Administration has expressed its interest in the renovation of the public transport service on that area in according with the raising demand for that service from the employees too – estimated for more than the $10\%$ of the total manpower.

After an in-depth analysis of the business park area, its current public transport service’s offer and the “new” employee’s mobility demand needs, we will focus on the possible ways to match them proposing new public transport solutions, acting both on the actual line side and studying new ones. All these new proposals will be analyzed from the top planning level –lines features- until the ordinary daily operation detail – sizing and scheduling of the service- in a consistent way, by the usage of a technical and an economic analysis to ensure the service overall sustainability. The last step will take into consideration the market launch strategy to find effective ways to promote and diffuse the new service awareness and its benefit among all the business park’s employees. 
